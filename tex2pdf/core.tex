\documentclass[12pt]{article}
\usepackage[twoside,
	paperwidth=13cm,
	paperheight=19.7cm,
	includehead,
	textheight=15.85cm,
	textwidth=10.2cm,
	outer=1.5cm,
	top=1.1cm,
	headheight=0.5cm,
	headsep=0.7cm,
	marginparwidth=0pt,
	marginparsep=0pt,
]{geometry}
\newbox\voidbox
\newdimen\columnruleht
\newdimen\colsgap
\colsgap=5mm

\def\doublecolumnsetup{
	\dimen0=\textwidth
	\advance\dimen0 by -\colsgap
	\divide\dimen0 by 2
	\hsize=\dimen0
	\hbadness=10000
	\updatecolumnruleht
}

\def\updatecolumnruleht{\begingroup\regularfont
	\columnruleht=-\topskip
	\advance\columnruleht by 1ex
	\global\columnruleht=\columnruleht
	\endgroup}

\newdimen\freespace
\def\updatefreespace{\ifdim\pagetotal>\textheight
		\freespace=2\textheight
	\else{\output={\unvbox255\penalty0}\eject}
		\freespace=\textheight\fi
	\advance\freespace by -\pagetotal
	\advance\freespace by -\pagedepth}

\def\thinspace{\nobreak\hskip2pt}

\newlanguage\lalang
\language\lalang
\input source/la-hyphenation.dic

\newlanguage\frlang
\language\frlang
\input hyph-fr.tex

\baselineskip=4mm plus 0pt
\parskip=4mm plus 2.5mm minus 1mm
%\clubpenalty=1000
%\widowpenalty=1000

\usepackage{microtype}
\usepackage{needspace}
\usepackage{lettrine}
\usepackage[document]{ragged2e}
\usepackage{setspace}
\usepackage{multicol}
\usepackage{paracol}
\usepackage{fancyhdr}
\usepackage{pdfpages}
\usepackage{enumitem}
\usepackage{mdframed}
\usepackage{nowidow}
%\usepackage{lua-visual-debug}
\usepackage{hyperref}
\usepackage{xfakebold}

%\input{source/hyphens}
\input adapter/titles
\input adapter/paragraphs
\input adapter/cantus
\input adapter/psalterium
\input adapter/strings
%\input adapter/tables

\newdimen\capheight%
\newdimen\descentdepth%

\newcommand{\hstyle}[1]{\fontsize{14pt}{14pt}\decorativ #1}

\pagestyle{fancy}
\fancyhead{}
\fancyhead[RO,LE]{\hstyle{\thepage}}
\fancyhead[CE]{\hstyle{\dayTitleShort}}
\fancyhead[CO]{\hstyle{\officeTitleShort}}
\renewcommand{\headrulewidth}{0pt}
\fancyfoot{}

\setlist{itemsep=0ex, parsep=0pt, topsep=0pt}

\setlength\columnsep{5mm}
\setlength\columnseprule{0.4pt}
\renewcommand{\multicolmindepthstring}{}
\setlength\multicolsep{0pt}
\setlength{\splittopskip}{0pt}
\setlength\premulticols{0pt}

\newenvironment{body}{
	\fontdimen2\font=1ex
	\fontdimen3\font=0.8ex
	\fontdimen4\font=0.5ex
	\pretolerance=-1

	\begin{justify}
		\flushbottom%
		\bodyStyle%
		\settoheight{\capheight}{ABCDEFGHIJKLMNOPQRST}%
		\settodepth{\descentdepth}{pqg}%
		\fontsize{11pt}{5mm}\selectfont%
		}{
	\end{justify}
}

\def\bodyStyle{
	\setlength{\parsep}{0pt}
	\setlength{\parskip}{4mm plus 2mm minus 1mm}
	\setlength{\lineskip}{0pt}
	\setlength{\lineskiplimit}{0pt}
	\setlength{\baselineskip}{5mm plus 0pt minus 0pt}
	\setlength\topskip{8.5pt}%
	\setlength{\parindent}{4mm}
	%	\setlength{\ttradFontSize}{10pt}
	%	\setlength{\ttradLineSkip}{4mm plus 0pt minus 0pt}
	%\setlength{\gregoLineSkip}{1cm plus 1mm minus0pt}
}

\def\psalteriumStyle{
	\setlength{\parskip}{2mm}%
}

\setlength{\ttradFontSize}{9pt}


%\showoutput
%\showboxdepth=3


\begin{document}
\begin{body}
	\UseRawInputEncoding{\documentclass[11pt]{extarticle}
\usepackage{atbegshi}
\usepackage{pdfpages}
\usepackage[paperwidth=10.2cm, paperheight=3cm, margin=0mm]{geometry}
\usepackage{gregoriotex}
\usepackage{twoopt}
%\usepackage{lua-visual-debug}

\newbox\voidbox
\newdimen\columnruleht
\newdimen\colsgap
\colsgap=5mm

\def\doublecolumnsetup{
	\dimen0=\textwidth
	\advance\dimen0 by -\colsgap
	\divide\dimen0 by 2
	\hsize=\dimen0
	\hbadness=10000
	\updatecolumnruleht
}

\def\updatecolumnruleht{\begingroup\regularfont
	\columnruleht=-\topskip
	\advance\columnruleht by 1ex
	\global\columnruleht=\columnruleht
	\endgroup}

\newdimen\freespace
\def\updatefreespace{\ifdim\pagetotal>\textheight
		\freespace=2\textheight
	\else{\output={\unvbox255\penalty0}\eject}
		\freespace=\textheight\fi
	\advance\freespace by -\pagetotal
	\advance\freespace by -\pagedepth}

\def\thinspace{\nobreak\hskip2pt}

\newlanguage\lalang
\language\lalang
\input source/la-hyphenation.dic

\newlanguage\frlang
\language\frlang
\input hyph-fr.tex

\baselineskip=4mm plus 0pt
\parskip=4mm plus 2.5mm minus 1mm
%\clubpenalty=1000
%\widowpenalty=1000

\setmainfont{PlantinStd-Light}[
	BoldFont=PlantinStd-SemiBold,
	ItalicFont=PlantinStd-LightItalic,
	BoldItalicFont=PlantinStd-BoldItalic,
	SmallCapsFont=Sabon,
	SmallCapsFeatures={Letters=SmallCaps},
	Ligatures=TeX
]
\newfontfamily\decorativ[
	Numbers=OldStyle,
	WordSpace=1.7,
	ItalicFont=SabonItalic,
	Ligatures=TeX,
	FakeBold=1.3
]{Sabon}
\newfontfamily\lettrine{Sabon}
\newfontfamily\courant[Ligatures=TeX]{PlantinStd-Light}

\fontsize{11pt}{5mm}\courant

\newdimen\gregoLineDepth
\gregoLineDepth=3mm

\makeatletter
\renewcommand{\strut}[1][.3\baselineskip]{%
	\hbox{%
		\vrule\@height\z@%
		\@depth#1%
		\@width\z@%
	}%
}
\makeatother

\newunicodechar{℣}{\mbox{V\kern-0.85em\reflectbox{\rotatebox[origin=c]{25}{∫}}\kern-0.05em}}
\newunicodechar{℟}{\mbox{R\kern-0.5em\reflectbox{\rotatebox[origin=c]{30}{∫}}\kern-0.25em}}
\newunicodechar{ǽ}{æ\hbox to 0pt{\kern-0.5em´\hss}}

\gresetspecial{R/}{℟}
\gresetspecial{V/}{℣}
\renewcommand{\GreStar}{\courant *}
\renewcommand{\GreDagger}{\courant †}
\renewcommand{\Rbar}{℟}

\grechangestaffsize{15}
\grechangedim{baselineskip}{\the\paperheight}{fixed}
\grechangedim{intersyllablespacenotes}{0.25cm}{fixed}
\grechangedim{minimalspaceatlinebeginning}{0.45cm}{fixed}
\grechangedim{spaceaftersigns}{0.1cm}{fixed}
\grechangedim{spacebeforeinlinecustos}{0.1cm}{fixed}
\grechangedim{spacebeforeeolcustos}{0.4cm}{fixed}
\grechangedim{spacelinestext}{0.555cm}{fixed}
\grechangedim{annotationseparation}{1pt}{fixed}
\grechangedim{spacebeneathtext}{\the\gregoLineDepth}{fixed}

\gresetclivisalignment{never}

\grechangestyle{modeline}{\scshape}
\grechangestyle{annotation}{\decorativ\fontsize{10pt}{0pt}\selectfont}

\grechangedim{spacelinestext}{4.5mm}{fixed}
\gresetnoteadditionalspacelinestext{manual}
\grechangedim{noteadditionalspacelinestext}{2.3pt}{fixed}
\grechangecount{noteadditionalspacelinestextthreshold}{4}%
\gresetcustosalteration{invisible}

\directlua{%
    internalListOfPageHeight = {}%
}

\newcommand{\enablePageCropping}{
    \directlua{
        local function post_linebreak(current)
            while current do
                if current.id == node.id("hlist") then
                    local height = current.height
                    local depth = current.depth
                    local total_height = (height + depth) .. "sp"
                    table.insert(internalListOfPageHeight, total_height)
                end
                current = current.next
            end
            return true
        end
        luatexbase.add_to_callback('post_linebreak_filter', post_linebreak, 'Hook after line break')
    }
}

\newcommand{\getHeightOfPage}[1]{\directlua{%
    local height = internalListOfPageHeight[#1]
    if height == nil then
        tex.print("\the\paperheight")
    else
        tex.print(height)%
    end
}}

\makeatletter
\newcommand{\convert}[1]{\strip@pt\dimexpr0.996264009963#1\relax}
\makeatother

\makeatletter
\newdimen{\new@pageheight}
\newdimen{\given@pageheight}
\AtBeginShipout{%
    \setlength{\given@pageheight}{\getHeightOfPage{\thepage}}%
    \new@pageheight=\dimexpr\pagegoal-\given@pageheight\relax%
    \edef\mypdfpageattr{/CropBox [0 \convert{\new@pageheight} \convert{\paperwidth} \convert{\paperheight}]}%
    \expandafter\pdfpageattr\expandafter{\mypdfpageattr}%
}
\makeatother

\makeatletter
\renewcommand{\strut}[1][.3\baselineskip]{%
    \hbox{%
        \vrule\@height\z@%
        \@depth#1%
        \@width\z@%
    }%
}
\makeatother

\let\oldGreSetThisSyllable\GreSetThisSyllable
\renewcommand{\GreSetThisSyllable}[5]{\oldGreSetThisSyllable{#1\strut[\gregoLineDepth]}{#2}{#3}{#4}{#5}}

\directlua{pwd = "\CurrentFilePath"}
\newcommand{\checkSyllabs}[1]{\directlua{
local cmd = [[echo "#1"]] .. ' | php "' .. pwd .. '/check-syllabs.php"'
local status, msg = os.execute(cmd)
    if (status == nil) then
        print("--shell-escape must be active to check syllabification")
    end
}}


\makeatletter
\long\def\nabcsnippet#1{%
	\directlua{gregoriotex.direct_gabc("\luatexluaescapestring{\unexpanded\expandafter{#1}}", "nabc-lines:1;", \gre@allowdeprecated@asboolean)}%
}%
\makeatother

\newcommand{\snippet}[1]{%
	\checkSyllabs{#1}%
	\gabcsnippet{#1}%
}
\newcommand{\enableNabc}{%
	\renewcommand{\snippet}[1]{%
		\checkSyllabs{##1}%
		\nabcsnippet{##1}%
	}
}

\let\oldGreannotation\greannotation
\renewcommand{\greannotation}[1]{%
	\oldGreannotation{%
		{\vrule width0pt height2.8mm depth1.2mm}%
		#1%
	}%
}

\enablePageCropping

\newcommand{\gabcDisplay}[1]{
	\grechangestyle{initial}{\fontsize{90}{90}\selectfont\decorativ}
	\snippet{#1}
}

\newcommand{\gabcNormal}[1]{
	\grechangestyle{initial}{\fontsize{35}{35}\selectfont\decorativ}
	\gresetinitiallines{1}
	\snippet{#1}
}

\newcommand{\gabcAnnotation}[3]{
	\greannotation{#1}
	\greannotation{#2}
	\grechangedim{annotationraise}{1mm}{fixed}
	\gabcNormal{#3}
}

\newcommand{\gabcSimplex}[1]{
	\gresetinitiallines{0}
	\snippet{#1}
}

\newcommandtwoopt{\gabcAnt}[3][Anti-][phona]{
	\greannotation{#1}%
	\greannotation{#2}%
	\grechangedim{annotationraise}{-3mm}{scalable}%
	\gresetinitiallines{1}%
	\greillumination{\null}%
	\snippet{#3}%
}
}
\end{body}
\end{document}
